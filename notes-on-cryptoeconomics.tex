\documentclass[12pt]{report}

\usepackage{hyperref}

\newcommand{\link}[2]{\href{#1}{#2}}

\begin{document}

\title{Notes on cryptoeconomics}
\author{Oskar Thoren\thanks{Email: ot@oskarthoren.com}$^{,}$\thanks{Web: http://oskarth.com}}

\maketitle

\chapter{Prelude}

Notes on cryptoeconomics and related subjects. Started September 9, 2018.

With a focus towards specific problems that I want to solve. These problems
should have more precise problem statement. A rough outline will do for now.
These are specific problems that are meaningful and relevant and guide reading,
even if more general tools are desirable and there is other problems and stuff
that is useful.

Additionally, specific papers and free recall notes of these. Then we'll see
where a synthesis makes sense.

Also roughly answer things like: what do you know and what would be useful to
know?

Previously did a write-up of A Layman's Introduction to Cryptoeconomics.

\textbf{Questions}:

1) Incentivized Whisper nodes, including mail server. General problem: full node
incentivization and running without a cluster or servers at all.

See things like: Miners; rent proposal; full node incentivization, as well as
altruistic models (scale arguments, E.O. Wilson tangent?). Also worth checking
re BT and Tor nodes, etc. Any form of p2p.

Litmus test: A system survive without a cluster or subsidised servers.

2) DAO and compensation, funding of public goods and the likes.

Should probable be phrased as question. Finding a good question statement is
useful. See Polya for examples of ways of thinking structurally about these
problems.

Litmus test: organization/community survive without core contributors/paycheck.

\chapter{Readings}

Anything by Vitalik or Szabo is a good start and meaty. Have a bunch in inbox.
Meta: what specific ways of reading is useful? Something like the following to
keep in mind:

- Who are the actors in this system?
- What's the set of rewards and punishments?
- What tools are used and introduced?
- Is there any empirical basis for this or is it still a theoretical construct?
- Lindy effect: Comparable that's 30y/10y or so old? Otherwise maybe BS.


TODO: Set up basic reference system
